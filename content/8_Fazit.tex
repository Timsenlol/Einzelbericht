\chapter{Fazit und Reflexion}\label{app:Sprints}
Wir sind mit dem Ziel in das Projekt gestartet, unser aus den Vorlesungen erlangtes Wissen in einem realistischen praktischen Projekt zu vertiefen. Neue Methoden, Arbeitsweisen und motivierte Kommilitonen und Technologien kennenzulernen motivierten mich zu Beginn dieses Projekt ebenfalls sehr. Die größte Herausforderung im Rahmen des Projektes war es, das Zeitmanagement zwischen meiner regulären Arbeit, dem Lernen für das Studium und das Projekt miteinander zu verbinden. Als sehr anregend und konstruktiv empfand ich die Zusammenarbeit und die kurzen Kommunikationswege in der Projektgruppe. Die offene und freundschaftliche Atmosphäre schaffte optimale Rahmenbedienungen für die Umsetzung des Projekt. Was ich als besonders wertschöpfend wahrgenommen habe, waren unsere regelmäßigen Code-Reviews. Vieles von dem dort Gelernten konnte ich in meinem beruflichen Alltag anwenden. Ein gutes Beispiel hierfür sind die "Spring Security Flows", die Tom uns vorgestellt hatte.  Der Aufbau des Projektes von der Formulierung der Userstories bis zu der Implementierung der finalen End-to-End-Tests haben dem Projekt eine robuste Struktur gegeben. Mit Stolz erfüllt mich, dass die Userstories und Ablaufdiagramme, die wir in den ersten Sprints formuliert, gezeichnet und dann kontinuierlich verbessert haben, letztendlich sehr nah an dem tatsächlichen Produkt sind. Die genutzten Methodiken versuche ich seitdem in meinen Arbeitsalltag mit einzubringen. Besonders gewachsen bin ich an der Ausarbeitung einer Teststrategie mit der anschließen Implementierung der übergreifenden Tests sowie an den Beteiligungen an den architektonischen Entscheidungen. Ein zentraler Punkt war hierbei die Einführung von Contract-Tests (Pact), um die Stabilität unser Schnittstellen im produktiven Betrieb zu gewährleisten und "Breakingchanges" frühzeitig zu erkennen. Diskussionen über den Aufbau der Services, die KI-Integration oder den Umgang mit blockierenden synchronen Aufrufen wurden stets konstruktiv und kritisch geführt, was sich als überaus wertschöpfend herausgestellt hat. Die Entscheidung auf Virtual Threads zu setzen, um ressourceneffizient an der KI-Schnittstelle zu bleiben, war für mich technisch eine besondere Bereicherung, da es das Projekt in meinen Augen auf ein höheres qualitatives Niveau gehoben hat. Wie ich bereites in der Einleitung beschrieben hatte, empfand ich die technische Auseinandersetzung mit der KI als sehr wertvoll. Das Projekte schaffte einen optimalen Rahmen, um sich mit dieser in der Zukunft wichtigen Technologie auseinanderzusetzen. Das ich für ein Spring-Backend Kotlin nutzte statt Java brachte mir auch sehr viel. In meinem Arbeitsalltag nutze ich seitdem immer öfter auch Kotlin, da es in meinen Augen die bessere Programmiersprache ist.