\chapter{Sprints}\label{app:Sprints}
\section{Sprint 0 24.01.2025 bis 21.02.2025}
Nach der Zusage der Verwaltung, dass wir das Projekt in der urspürnglich von uns beantragten Gruppe durchführen zu dürfen, setzten wir uns zeitnah mit Dr. Kötting in Verbindung. Hinsichtlich des Themas hatten wir noch Bedenken, da Dr Köttings initiale Projektidee, die er auch in der ersten Informationsveranstaltung dem Studiengang vorgestellt hatte uns allen nicht gefallen hatte. Im ersten Gesprächen stellte sich allerdings raus das Herr Dr. Kötting hier offen war das Thema anzupassen und daher beauftragte er uns mit der Ausarbeitung einer neuen Projektidee. Ihm war es wichtig, dass die ausgearbeitete Projektidee, Fragestellungen oder Probleme mit Hilfe von KI lösen kann.
Das Projektteam erarbeitete sich die Projektidee im Rahmen einer strukturierten Kreativitätsphase.
Hier wurde das "1-3-All"-Verfahren genutzt (angelehnt an das "1-2-4-All" Verfahren 
\cite{Desginthinking124}), um eine große Menge an Ideen zu entwickeln. Nach der Sammlung der Ideen,  stellte jede Person Ihre Ideen der Gruppe (3er Gruppe oder alle) vor, wo diese dann im Kollektiv diskutiert, verfeinert und gegeben falls zu mit anderen Ideen geclustert wurde. Abschließend hatte jeder Teilnehmer drei Stimmen um sie auf die erarbeiteten Ideen zu verteilen. Die drei Ideen mit den meisten Stimmen würden nochmal unsern Auftraggebern Dirk Kötting vorgestellt und gemeinsam diskutiert. Wichtige Faktoren waren hier der erwartbare Mehrwert für einen potenziellen Kunden und die technische Umsetzbarkeit.
Letztendlich hat sich die Projektgruppe für das Security Tool entschieden. Ausschlaggebend waren zum einem die Interessen und Kenntnisse unseres Auftraggeber, von dem dieses Projekt profitiert hat. Zum andern das dieses Projekt eine Lösungen für echte Probleme von mittelständischen Unternehmen anbietet. 
In einem anschließenden Treffen haben wir uns ein ein Codeverwaltung- und Projektmanagement-Tool geeinigt.  Für die Codeverwaltung fiel unsere Wahl auf Github, als eine der führenden Codeverwaltungsplattformen. Github ist grundsätzlich kostenlos. Ein weiterer Faktor waren die "GitHub Student Developer Packs" mit denen wir einen kostenlosen Zugang zu erweiterten Funktionen und kostenpflichtigen Angeboten erhielten (vgl \cite{github_student_pack}). 
Beim Projektmanagementtool fiel unsere Wahl auf Jira. Auch Jira ist ein im Mark etabliertes Tool im Bereich des Projektmanagement. Wir nutzen Jira zum verwalten unser Vorhaben in Form von Epics, darunter clusterten wir auch später die die jeweiligen Task, die wir in dem entsprechenden Sprint bearbeiten wollten. Für die Übersicht haben wir ein Kanban Board angelegt, wo die entsprechenden Vorhaben mit ihrem aktuellen Status für jedes Gruppenmitglied jederzeit über ihre eigenen Jira Zugänge einsehbar war. Ein wichtiger Punkt der für Jira gesprochen hatte war, dass unser Product Owner Christian Langer breites viel Erfahrung im Umgang mit Jira hatte. Er das Wissen über den Umgang mit dem Tool ins Projekt tragen.  Im nächsten Schritt habe ich mich um meine lokalen Entwicklungsumgebungen gekümmert. Da ich bereits viel Erfahrung mit den Entwicklungsumgebungen von Jetbrains hatte, wollte ich wieder auf diese setzen. Nach Recherche bin ich auf "Jetbrains Student Pack" gestoßen, welches Studenten erlaubt alle kostenpflichtigen Entwicklungsumgebungen kostenlos zu nutzen (vgl. \cite{jetbrains_student}). Initial installierte ich PyCharm und Intellij.
Innerhalb unser einzeln Workstreams haben wir nun die Anforderungen an die jeweiligen Module herausgearbeitet. Hierbei haben wir versucht aus den groben Epics, auf die wir uns mit Herrn Dr Kötting committet hatten, einzelne Anforderungen abzuleiten. Christian, Daniel und ich haben für das Phishing Modul im ersten Schritt ein grobes Ablaufdiagramm gezeichnet, um anschließend draus die einzelnen Anforderungen leichter erarbeiten zu können. Abschließend haben in der gesamten Projektgruppe die ausgearbeiteten Anforderungen vorgestellt und besprochen. Hier haben wir nochmal einige Anpassungen vorgenommen zum Beispiel hinsichtlich der zu auszuwertenden Parameter wie "Schwierigkeitsgrad". Diese Einstellung haben wir noch einigen Diskussionen zunächst verworfen.
Nachdem wir alle zufrieden mit den formulierten Anforderungen waren, haben wir diese auch nochmal unsern Auftraggeber vorgestellt um sicherzugehen, dass diese Anforderungen seinen Vorstellungen entsprechen. Dr. Dirk Kötting hat hier nochmal sehr gute Einwürfe gebracht und wir konnten die initialen Anforderungen nochmal verfeinern. Ein wichtige Anmerkung, war der Datenschutzgedanke und der Gedanke der Rechtssicherheit. Draus haben haben wir zum Beispiel die Anforderung abgeleitet, dass der Anwender vor der erstmaligen Nutzung des Moduls eine Nutzungsbedingung akzeptieren muss.

\begin{figure}[H]
    \centering
    \includegraphics[width=0.8\textwidth]{figures/erster_Wurf_Ablauf.png} 
    \caption{Erster Wurf für den Ablauf des Phishing-Workflows}
    \label{fig:architektur1}
\end{figure}
\section{Sprint 1 21.02.2025 bis 21.03.2025}
In diesem Sprint haben wir uns mit der grundsätzlichen Architektur innerhalb des Phishing Moduls beschäftigt. Zunächst haben wir uns gemeinsam auf die Strategie geeinigt eine eigene  KI zu schreiben die Phishingmails geniert. Unter dieser Bedingung haben wir geplant ein Spring Backend zu schreiben, welches orchestrierende Aufgaben übernimmt sowie ein Python Flask Backend, welches die KI selbst aufruft.
\begin{figure}[H]
    \centering
    % Hier wird der Pfad zum Ordner "figures" angegeben
    \includegraphics[width=0.7\textwidth]{figures/EntwurfSprint1.png} 
    \caption{Entwurf der Modul Architektur im Sprint 1}
    \label{fig:architektur}
\end{figure}
Im ersten operativen Schritt habe ich das Grundgerüst des Springbootprojekts angelegt.Zu Erstellung des Projektes habe ich der Spring Initializr. Dieses kostenlose Tool ermöglicht eine geführte Erstellung der Projektstruktur. Nach Rücksprache mit Christian und Daniel haben wir uns für ein Maven Projekt in Kotlin entschieden. Maven.Die Punkte die für Maven gesprochen haben sind zum einem seine Leichtgewichtigkeit sowie meine Praxiserfahrung mit dem Buildtool. Dadurch konnte ich Christian und Daniel effizent in das Tool einarbeiten. Für Kotlin habe ich mich bewusst entschieden, denn ich wollte die Sprache im Rahmen des Projektes zu evaluieren und zu prüfen ob sie sich  eignet in meinen  Arbeitsprojekten eingesetzt zu werden. Kotlin ist eine deutlich moderene Sprache als Java und bietet einige Vorteile von denen wir auch im Rahmen unsers Projekts profitiert haben. 
\begin{itemize}
  \item \textbf{Null Saftey}: Variablen müssen explizit als "nullbar" deklariert werden. Das mindert die Gefahr von "NullPointerExceptions" erheblich. Schon während der Entwicklungsphase, hat sich dieses Feature als äußert wertvoll erwiesen. Zum einem haben wir nicht viele solcher Exceptions gehabt und wenn wir welche hatten kamen nur nullbare Variablen in Frage.
  \item \textbf{Kompakter Code}:Bolier plate Code wird drastisch reduziert. Wir konnten sehr viele Zeilen an Code sparen allein dadurch, dass Kotlin hier einfach modrner ist als Java. Zum Beispiel wird das arbeiten auf Listen deutlich vereinfacht. Wo man in Java mit einer externe Libary oder vielen If Else arbeiten muss, kann man in Kotlin auch komplexere Sachen in 2-3 Zeilen abbilden. 
\end{itemize}
Wir haben uns darüber hinaus noch auf eine Branchstrategie geeinigt (siehe 4.2 Branching Vorgehen):
Grundsätzlich wurde für jedes in der Entwicklung befindende Feature, ein Feature-Branch angelegt. Nach Fertigstellung und Code Review durch die Gruppe, wurde dieser Branch dann über einen Pull Request in den in unsern Main-Branch integriert. Darüber hinaus habe ich im aktuellen Sprin eine Github Action gebaut. Die Github Action ist so gebaut dass diese bei jedem Push oder Pull Request ausgelöst wird. Sie ist so gebaut, dass der Projekt mit Maven gebaut wird und alle Tests ausgeführt werden. Github Action waren für uns dank des "Github Student Developer Packs" kosteneffizient nutzbar für uns. 

