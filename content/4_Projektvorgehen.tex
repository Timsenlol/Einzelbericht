\chapter{Projektvorgehen}\label{ch:outlook}

\section{Abgewandelter Scrum-Prozess}\label{ch:outlook}
Wir setzten bewusst auf das Scrum-Framework, passten es allerdings an unsere Rahmenbedingungen an.
\subsection{Springlänge}
Da unser gesamtes Team berufstätig ist, war ein Treffen Präsenz mit allen 7 Teammitgliedern hauptsächlich in der Blockveranstaltungen der Hochschule möglich. Daher wurde die Sprintlänge auf die Abstände zwischen den Vorlesungsblöcken gelegt. In der Regel waren das 4-8 Wochen. Für uns war es wichtig Review, Planning und Retrospektive vor Ort gemeinsam durchzuführen. 
\subsection{Backlog}
Das Prodocut Backlog wurde gemeinsam mit dem Auftraggeber Dr. Dirk Kötting abgesimmt und fokussierte sich auf die beiden Hauptmodule Phishing-Modul und CVE-Modul. Die war im Backlogs als Epics darstellten. Die Workstreams formulierten für ihre Module, die Userstories weitestgehend selbst und stellten diese dann anschließend in einem Review vor. 

