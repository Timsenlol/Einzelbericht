\chapter{Einleitung}\label{ch:Einleitung}
\section{Projektkontext}\label{ch:Projektkontext}
Mit dem Ziel, die theoretischen Inhalte aus den Vorlesungen in einem praktischen Projekt anzuwenden und zu vertiefen, entstand das vorliegende Projekt "SecHub", welches mich über mehre Semester hinweg begleitet hat. Ein besonders Augenmerk wurde auf die Vorgehensmodelle rund um das Scrum-Framework gelegt. 
Die Ausrichtung des Projekts lag auf der Entwicklung eines KI-gestützten Systems, das eine Organisation bei der Erkennung, Simulation und Kommunikation sicherheitsrelevanter Schwachstellen unterstützen soll. Unser externer Auftraggeber, der unser Projekt begleitet hat, Dr. Dirk Kötting gilt als erfahrener Experte für KI- und IT-Security-Fragestellungen. Durch seine Expertise konnten wir neue Einblicke in das Thema gewinnen.
Neben dem Softwareprodukt selbst stand über den gesamten Projektzeitraum auch die Zusammenarbeit als Team im Fokus. Wir mussten ein gemeinsames Verständnis für die Problemstellung entwickeln, uns organisieren und die gelernten Projektmanagementprinzipen gezielt einsetzen. 
\section{Motivation}\label{ch:Motivation}
Besonders motivierend war für mich die technische Auseinandersetzung mit KI. Auch wenn es vieles am Ende nicht ins fertige Produkt geschafft hat, war das Ausprobieren von verschiedenen Ansätzen in diesem Bereich besonders lehrreich und spannend für mich. Darüber hinaus hat mich die Herausforderung, dieses Projekt als Gruppe zu schaffen, stark motiviert. Die Möglichkeit neue Technologien, die ich in meinen Beruf in der Regel nicht verwenden kann, waren ein zusätzlicher motivierender Anreiz. 
\section{Rahmenbedingungen}\label{ch:Rahmenbedingungen}
Folgende Rahmenbedingungen hatten Einfluss auf das Projekt:
\begin{itemize}
  \item \textbf{Zeitliche Struktur} Im Dezember 2024 begann die Themenfindung. Die Implementierungsphase, die im Rahmen von mehreren Sprints durchgeführt wurde, wurde dann durch die Präsentationen im Februar 2026 abgeschlossen.
  \item \textbf{Methodik der Projektdurchführung} Die Durchführung des Projektes unterlag zentralen Scrum-Elementen. Beispiele hierfür sind die Rolleneinteilungen in Produktowner, Scrummaster, Softwareentwickler oder die regelmäßigen Sprintplannings und Reviews. 
  \item \textbf{Technologie} Vorgabe durch Dirk Kötting war, dass das finale Produkt  KI enthalten sollte.
  \item \textbf{Finanzelle Ressourcen} Durch den universitären Kontext standen nur geringe bis keine finanziellen Mittel für dieses Projekt zur Verfügung. Dieser Umstand hatte Einfluss auf die technologischen Entscheidungen.
  \item \textbf{Teamgröße} Mit sieben Personen waren wir für ein Scrum-Team ein relativ großes Team. Was auch eine gewisse Herausforderung an unsere Teamorganisation war. 
  \item \textbf{Berufsbegleitens Projekt}  Dadurch dass alle Teammitglieder in ihren jeweiligen Vollzeitjobs intellektuell stark eingebunden sind, erforderte das Projekt an hohes Maß an Flexibilität und Effizienz von uns.
\end{itemize}
\section{Ziel}\label{ch:Ziel}
In dem Projekt konnten wir verschiedene Ziele identifizieren:
\begin{itemize}
  \item \textbf{Produktziele:} Ziele die sich auf das Secuirtytool als solches beziehen: 
  \begin{itemize}
  \item Integration von KI Ansätzen in das Produkt
  \item Modul zur Aufbereitung, Verwaltung und zum Versand von Phishingmails  
  \item Modul zur Analyse und Aufbereitung von CVE-Schwachstellen
  \item Aufbau eines skalierbaren Backends
\end{itemize}
  \item \textbf{Persönliche Ziele}
    \begin{itemize}
  \item Aufbau von  Kompetenzen im Umgang mit KI
  \item Aufbau von technischen Kompetenzen im Bezug auf Softwarearchitektur und moderne Backendtechnologien  
  \item Aufbau von Wissen und Kompetenzen im Umgang mit Scrum
\end{itemize}
  
\end{itemize}
