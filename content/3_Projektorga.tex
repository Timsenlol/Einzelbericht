\chapter{Projektorganisation}\label{ch:orga}

\section{Rollenverteilung}\label{ch:orga}
Die Grundlage zur Durchführung dieses Projektes war Scrum. Dementsprechend wurden in der Projektgruppe die drei klassischen Scrum-Rollen Product Owner, Scrum Master und Developer identifiziert (vgl. \cite{scrumguide2020}). Der Projektgruppe war eine transparente Rollenwahl wichtig. Im Rahmen eines Austausch stellte jedes Gruppenmitglied seine Wunschrolle und Backuprolle vor und begründete seine Wahl mit der Vorstellung seiner individuellen Kompetenzen und Lernziele.
Die Projektgruppe konnte sich nach erregenden Austausch auf folgende Rollenverteilung einigen:
  \begin{itemize}
  \item \textbf{Product Owner}: Christian
  \item \textbf{Scrum Master}: Florian 
  \item \textbf{Development Team}: Stephan, Daniel, \textbf{Tim}, Tom, Sebastian
\end{itemize}
Anmerkt sei an dieser Stelle, dass der Scrum Master und Product Owner auch neben ihren Aufgaben aus der Rolle sich aktiv in der Entwicklung beteiligten. 
Zur Steigerung der Flexibilität und Effizienz haben wir uns ins Workstreams aufgeteilt. Die Aufteilung sag wie folgt aus:
\begin{itemize} \item \textbf{Phishing-Mail-Team}: Christian, \textbf{Tim}, Daniel \item \textbf{CVE-Parser-Team}: Florian, Stephan \item \textbf{Security und Deployment}: Tom \item \textbf{Frontend}: Sebastian \end{itemize}
Bei den Modulen war es uns wichtig, dass immer nach dem Vier-Augen-Prinzip von mindestens 2 Personen betreut werden, damit wir keine Wissenssilos aufbauen. Durch regelmäßigen Austausch hatten alle unsere Teammitglieder, trotz dieser Spezialisierung stets ein grundsätzliches Verständnis um die gesamte Codebasis und die eingesetzte Technik. 
\subsection{Persönliche Rolle im Projekt}\label{ch:orga}
Ich wählte initial die Rolle des Softwareentwicklers, da ich mich in dieser Rolle am wohlsten fühle und ich glaube, dass ich dem Team so am beste weiterhelfen konnte im diesem Projekt. Durch meinen beruflichen Hintergrund brachte ich ein gutes Know How in der Backendentwicklung mit Spring und .Net mit ins Team. Außerdem Kenntnisse über modere Testmethoden und Frameworks wie Pact oder Playwright. Des weiteren habe ich das Team mit Know How zu GitHub und CI-Pipelines unterstützt. Meine Stärke sehe ich darin technische Prozesse schnell zu verstehen, zu strukturieren und zu bei Bedarf zu verbessern.  
\section{Teamformung}\label{ch:orga}
Bereits im Vorfeld fand sich unsere Projektgruppe und teilte der Studiengangsverwaltung den Wunsch mit das Projekt gemeinsam durchführen zu wollen. Dieser Wunsch wurde bei der Gruppeneinteilung auch berücksichtigt, allerdings wurde die Gruppe initial um 7 Personen auf 14 Personen erweitert.  Schnell stellte sich heraus, dass ein effizientes und zielgerichtetes arbeiten, welches für das berufsbegleitende Projekt essenziell war praktisch nicht möglich war. Grundlegende organisatorische Aspekte wie Rollenklärung oder Terminfindung waren nicht realistisch umsetzbar.Eine agile Arbeitsweise lies sich in dieser Gruppengröße kaum realisieren. Nach intensiven Gesprächen mit den Projektcoaches  und der Verwaltung wurde die Rückkehr zu den ursprünglichen Siebener-Gruppe Gruppe genehmigt. 
\section{Pair und Mob Programming}\label{ch:orga}
Ein Großteil der Entwicklungsarbeit am Phishing-Mail-Modul fand im Rahmen von Mob- oder Pair-Programming statt. Als Workstream-Gruppe traf sich das Team häufig in Präsenz vor Ort in Nürnberg, um gemeinsam an dem Modul zu arbeiten. Dabei probierten wir unter anderem Konzepte wie Coding Dojos aus \cite{codingdojoorg}. Wie in der Fachliteratur beschrieben, führte das gemeinschaftliche Entwickeln \cite {williams2002pair, zuill2016mob} dazu, dass sich die Gesamtqualität des Codes spürbar verbesserte. Des Weiteren vertiefte sich dadurch unser gegenseitiges Verständnis für die eingesetzte Technik.
\section{Code Reviews}\label{ch:orga}
Ein uns besonderer wichtiger Punkt war, dass wir voneinander lernen wollten. Dies schafften wir indem wir regelmäßige Code Reviews durchführten. Hierbei stellte eine Gruppe oder Person seine letzten Changes vor und wir diskutieren diese im Kollektiv. Dieses Format erwies sich für uns als äußerst konstruktiv. Im Rahmen dieser Reviews wurden alternative Lösungsmöglichkeiten besprochen oder auch Architekturentscheidungen hinterfragt.