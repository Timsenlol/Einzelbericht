\chapter{Projektorganisation}\label{ch:orga}

\section{Rollenverteilung}\label{ch:orga}
Die Grundlage zur Durchführung dieses Projektes war Scrum. Dementsprechend wurden in der Projektgruppe die drei klassischen Scrum-Rollen Product Owner, Scrum Master und Developer identifiziert (vgl. \cite{scrumguide2020}). Der Projektgruppe war eine transparente Rollenwahl wichtig. Im Rahmen eines Austausch stellte jedes Gruppenmitglied seine Wunschrolle und Backuprolle vor und begründete seine Wahl mit der Vorstellung seiner individuellen Kompetenzen und Lernziele.
Die Projektgruppe konnte sich nach erregenden Austausch auf folgende Rollenverteilung einigen:
  \begin{itemize}
  \item \textbf{Product Owner}: Christian
  \item \textbf{Scrum Master}: Florian 
  \item \textbf{Development Team}: Stephan, Daniel, \textbf{Tim}, Tom, Sebastian
\end{itemize}
Wobei der Scrum Master als auch der Product Owner auch neben ihren Aufgaben aus der Rolle sich aktiv in der Entwicklung beteiligten. 
\subsection{Persönliche Rolle im Projekt}\label{ch:orga}
Ich wählte initial die Rolle des Softwareentwicklers, da ich mich in dieser Rolle am wohlsten fühle und ich glaube, dass ich dem Team so am beste weiterhelfen konnte im diesem Projekt. Durch meinen beruflichen Hintergrund brachte ich ein gutes Know How in der Backendentwicklung mit Spring und .Net mit ins Team. Außerdem Kenntnisse über modere Testmethoden und Frameworks wie Pact oder Playwright. Des weiteren habe ich das Team mit Know How zu GitHub und CI-Pipelines unterstützt. Meine Stärke sehe ich darin technische Prozesse schnell zu verstehen, zu strukturieren und zu bei Bedarf zu verbessern.  