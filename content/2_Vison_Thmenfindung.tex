\chapter{Themenfindung und Projektvision}\label{ch:vision}
\section{Themenfindung}\label{ch:vision}
Dr. Dirk Kötting der Auftraggeber, hatte zunächst kein Projekt für die Projektgruppe und beauftragte die Projektgruppe daher ihm eine Projektidee vorzustellen. Seine Vorgabe hierbei war, dass er nach einem Projekt sucht, welches Fragestellungen oder Probleme mit Hilfe von KI lösen kann.
Das Projektteam erarbeitete sich die Projektidee im Rahmen einer strukturierten Kreativitätsphase.
Hier wurde das "1-3-All"-Verfahren genutzt (angelehnt an das "1-2-4-All" Verfahren 
\cite{Desginthinking124}), um eine große Menge an Ideen zu entwickeln. Nach der Sammlung der Ideen,  stellte jede Person Ihre Ideen der Gruppe (3er Gruppe oder alle) vor, wo diese dann im Kollektiv diskutiert, verfeinert und gegeben falls zu mit anderen Ideen geclustert wurde. Abschließend hatte jeder Teilnehmer drei Stimmen um sie auf die erarbeiteten Ideen zu verteilen. Die drei Ideen mit den meisten Stimmen würden nochmal unsern Auftraggebern Dirk Kötting vorgestellt und gemeinsam diskutiert. Wichtige Faktoren waren hier der erwartbare Mehrwert für einen potenziellen Kunden und die technische Umsetzbarkeit.
Letztendlich hat sich die Projektgruppe für das Security Tool entschieden. Ausschlaggebend waren zum einem die Interessen und Kenntnisse unseres Auftraggeber, von dem dieses Projekt profitiert hat. Zum andern das dieses Projekt eine Lösungen für echte Probleme von mittelständischen Unternehmen anbietet.
\section{Projektvision}\label{ch:vision}
Auf Basis der Idee wurde von der Projektgruppe eine Vison für das Projekt erarbeitet. Das Kernprinzip der Anwendung sollte ein modularer Ansatz werden. Erdacht ist es, dass Sicherheitsfunktionen ohne größere Eingriffe hinzugefügt werden können.
Jede Sicherheitsfunktion wird in einem eigenen Modul abgebildet und ist grundsätzlich völlig unabhängig von andern Sicherheitsfunktionen im "SecHub". Die beiden ersten Module die die Projektgruppe im Rahmen des Projektzeitraums erarbeitet hat sind die \textbf{Phishing-Mail-Simulation} und das \textbf{CVE-Scanning mit KI gestützer Auswertung} 
\subsection{Phishing-Mail-Simulation}\label{ch:vision}
Ein Modul das einen vollständigen Workflow zur Erstellung, Verwaltung und Versand von Phishing-Mails anbieten soll. Ziele des Moduls sind:
  \begin{itemize}
  \item realistische Phishing-Mails zu erstellen
  \item Mitarbeiter zu sensibilisieren  
  \item detailliertes Reporting über das Klick-verhalten zur Verfügung zu stellen
\end{itemize}
\subsection{CVE-Scanning mit KI gestützer Auswertung}\label{ch:vision}
Dieses Modul soll sich mit der Analyse von Schwachstellen befassen. Ziele des Moduls sind:
  \begin{itemize}
  \item CVE Daten sammeln
  \item Schwachstellen auf Basis der Eingabe von Nutzers liefern können   
  \item Handlungsempfehlungen geben
\end{itemize}